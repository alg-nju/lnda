%--
\documentclass[UTF8]{ctexrep}


%-- layout
\usepackage[margin=1.2in]{geometry}

%--
\usepackage{url, hyperref}


%-- user-defined commands
\newcommand{\mybreak}{\vspace{1cm}}


%--
\begin{document}


%--
\title{\bf 分布式算法讲义}

%--
\author{黄\ \ 宇}

%--
\maketitle

%--
\tableofcontents


%--
\chapter{计算模型}



%--
\chapter*{素材}


YCSB \cite{Cooper10, YCSB-github}


\mybreak

计算模型的基础是抽象。首先介绍各种抽象。然后介绍由各种不同抽象,组合而来的各种模型 \cite{Cachin11}。


\mybreak

process groups,group membership。

virtual synchrony。

\cite{Bost10} (第六章)

Process groups are a powerful tool for the developer. They can have names, much like files, and this allows them to be treated like topics in a publish-subscribe system. 

One thinks of a process group as a kind of object (abstract data type), and the processes that join the group as importing a replica of that object. Virtual synchrony standardizes the handling of group membership: the system tracks group members, and informs members each time the membership changes, an event called a view change.



%-- bib
\bibliographystyle{acm}
\bibliography{lnda}


%--
\end{document}
