%--
\documentclass[UTF8]{ctexrep}


%-- layout
\usepackage[margin=1.2in]{geometry}

%--
\usepackage{url, hyperref}


%-- user-defined commands
\newcommand{\mybreak}{\vspace{1cm}}


%--
\begin{document}


%--
\title{\Huge \bf 分布式算法讲义}

%--
\author{黄\ \ 宇}

%--
\maketitle

%--
\tableofcontents

%--------------------------------------------------------------------------------
\part{计算模型}

%----------------------------------------
\chapter{分布式算法简介}

\cite{Barroso18}

多数据中心平台,从硬件设施,到软件基础设施(infrastructure)的介绍。


%----------------------------------------
\chapter{分布式计算模型}

计算模型的基础是抽象。首先介绍各种抽象。然后介绍由各种不同抽象,组合而来的各种模型 \cite{Cachin11}。


%--------------------------------------------------------------------------------
\part{消息传递算法}


%--------------------------------------------------------------------------------
\part{共享存储算法}


%--------------------------------------------------------------------------------
\part{进阶专题}


\chapter{专题1}


【virtual synchrony】

process groups,group membership。

virtual synchrony。

\cite{Bost10} (第六章)

Process groups are a powerful tool for the developer. They can have names, much like files, and this allows them to be treated like topics in a publish-subscribe system. 

One thinks of a process group as a kind of object (abstract data type), and the processes that join the group as importing a replica of that object. Virtual synchrony standardizes the handling of group membership: the system tracks group members, and informs members each time the membership changes, an event called a view change.

\mybreak

【混合BFT】

有些机器只会crash,不会叛变 \cite{Vukotic19}。

区块链领域也使用这个假设,来提升区块链共识的速度 \cite{Dang19}。

提供满足这种假设的off-the-shelf hardware systems,例如Intel的Software Guard Extensions (SGX) \cite{McKeen13}。


%--------------------------------------------------------------------------------
\part{实际案例}


%----------------------------------------
\chapter{案例1}

【系统类:cloud data store】

\cite{Bravo15}

对于cloud data store的介绍。

分布式系统中(主要是cloud data store中)对于ordering of events的tracking。弱一致系统,强一致系统中的clock的设计。


\mybreak

【工具类:TLA+】


\cite{TLA-Github}。



%--
\chapter*{素材}


YCSB \cite{Cooper10, YCSB-Github}













%-- bib
%\bibliographystyle{apalike}
%\bibliographystyle{acm}
\bibliographystyle{abbrv}
\bibliography{lnda}


%--
\end{document}
